% \cleardoublepage

\section{研究工作}

这一章主要介绍本篇文章的研究的几种工作的主要内容,各自的设计,以及各自实验的效果,并进行对比。当前模糊测试的主流是灰盒测试,如\cite{jeong2019razzer, chen2020muzz, xu2020krace, jiang2022context},这些工具利用灰盒测试,收集多线程程序运行过程中的线程交错信息,如别名覆盖率\cite{xu2020krace}等,来指导线程交错信息的变异。前面的主要都是基于变异的方式来进行模糊测试,但是这些工具有一个缺点就是无法生成一个比较好的初始输入,这会导致工具的探索效率随机性太高,可能在某个初始输入比较好的情况下,工具的性能非常高,在短时间内就可以探索到并发漏洞。但是如果初始输入不合适,可能花费大量时间也无法探索到能产生并发漏洞的线程交错情况,甚至根本无法找到,因为线程交错空间随着程序长度增加呈指数级上升,以当前的计算机能力几乎无法穷尽。

\subsection{RFF}

在\cite{wolff2024greybox}这篇论文中,Wolff等人通过将Reads-from Relation引入到模糊测试中来生成抽象调度,来代表一个等价调度集合,通过测试这个调度集合中的一个调度就可以代表整个集合。同时文章还将Reads-from作为新的反馈,利用这个度量引导模糊测试器进行后续的探索。下面将对RFF的相关核心思想进行介绍。

\subsubsection{概述}

当前用灰盒模糊测试检测并发漏洞的工具主要有以下几个特点。首先,灰盒模糊测试器保留一个调度的集合,这个集合中的每一项都是一个调度,可以用来对多线程程序进行测试。其次,调度集合是所有调度集合的一个子集,因此模糊测试器会在每一轮测试中从这个集合中选取一个最佳的项用于进行测试。为了量化调度之间优劣关系,模糊测试器就会通过代码覆盖率反馈来量化调度是否是足够好的,目前也有许多其他形式的反馈被提出来,用于更加准确地反馈多线程程序的交错程度。

尽管当前的处理已经能够发现许多由多线程程序引起并发漏洞,但是仍然还是有许多问题有待解决,他们对于提升模糊测试的效率有极大的影响