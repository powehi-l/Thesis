\cleardoublepage

\section{绪论}

\subsection{研究背景与意义}

随着计算机科技的不断发展,各种计算机技术已经成为人们生活不可或缺的一部分,聊天软件、移动支付、影音技术都为日常生活带来极大的便捷。但是,这些技术在带来便捷的同时也导致了越来越多的安全问题。2021年4月,超过5.3亿Facebook用户的数据被泄露,其中包括姓名、出生日期、社交关系等敏感信息。2022年6月,西北工业大学发布声明称遭受境外的网络攻击,导致被窃取部分敏感信息。2023年6月,苹果iOS操作系统被曝存在安全漏洞,可能导致用户手机被黑客完全控制。由此可见,网络安全已经成为不可忽视的问题,个人的隐私及安全由此受到严重的威胁。因此需要尽量减少计算机系统中的漏洞,在让人们享受便捷的同时能够保护个人隐私安全。

与此同时,由于摩尔定律逐渐接近极限,多核处理器由于其对性能的提升被广泛地使用。相应地,由于并发编程模式能够充分发挥多核处理器的性能,因此也被编程人员青睐,合理的并发编程可以的大大提升程序运行的效率。但是这也引入了和顺序程序不同的安全漏洞,当并发程序不正确地处理多个线程之间的同步时就会导致并发错误,进而导致并发漏洞,黑客可以通过这些漏洞对系统发起攻击。例如,当多个线程对同一个变量的访问没有受到保护时,由于运行过程中线程执行的序列不同,最后变量的值被错误地更改,导致漏洞的产生。

目前已经有多种并发漏洞检测手段,包括静态检测、符号执行、模糊测试等,但是静态分析存在着准确率低,误报率高的缺点;符号执行的开销则会随着程序复杂度的提升迅速增大,而模糊测试可以在保证准确率的同时以可接受的开销来挖掘漏洞。随着计算机性能的提升以及AFL-ref的提出,模糊测试成为挖掘漏洞的重要手段。模糊测试最早由Barton Miller提出,最初用来测试程序的健壮性,后来才逐渐用于进行漏洞挖掘,模糊测试技术经过多次重要的发展,现在已经逐渐完善并得到广泛应用。但是由于并发程序的复杂性以及运行时的不确定性,并发漏洞能够被触发的概率很小,导致检测并发漏洞并稳定复现仍然存在困难。

当前通用的模糊测试工具只能处理通用地程序输入,而不能扩展到程序并发执行的维度,尽管目前存在一些对并发程序进行模糊测试的工作,但是都存在一定的局限性。因此本文主要探究将模糊测试应用在并发漏洞挖掘上,以实现高效的并发漏洞挖掘。


\subsection{国内外研究现状}

\subsubsection{利用部分顺序规约进行并发漏洞挖掘}

并发程序中线程或者进程的执行可能以多种方式交织,如果直接进行探索会导致状态空间呈指数级增长,在有状态的情况下甚至会造成状态爆炸,这对于当前的计算能力来说是不现实的,计算机无法在合理时间内对如此大量的情况进行探索,因此有必要提出一种能够减少并发程序探索空间的技术。部分顺序规约作为一种检查各个状态和行为间的独立性以减小整体的状态空间的技术,正好可以应用到并发漏洞挖掘,通过识别关键数据的依赖关系,将等价的并发程序交织归约到同一个集合,每一个集合只需要执行一次就可以对集合中的所有情况进行检测。

早在2005年,Flanagan\cite{flanagan2005dynamic}等人通过动态地跟踪进程间的相互作用,识别出需要回溯的点,从而探索状态空间中的替代路径。最终结果显示动态的部分顺序规约可以在线程不断增多的情况下避免状态爆炸的产生。

2014年,Abdulla\cite{abdulla2014optimal}等人提出了一种新的动态部分顺序规约算法并实现工具optimal-DPOR,通过引入源集合记录当前某个调度点上的所有可能的操作,避免探索与已探索的执行等价的执行路径,从而减少冗余探索。同时这篇文章还引入了唤醒树机制,控制未来探索的初始步骤,确保算法不会遇到任何睡眠集阻塞的探索。最终实验结果显示,optimal-DPOR在导致不超过10\%开销的情况下加速了并发空间的探索。

2019年,Abdulla\cite{abdulla2019optimal}等人提出了首个针对顺序一致性语义下的多线程程序的无状态模型检查算法,算法基于一种更粗粒度的等价关系——reads-from,这种关系保留了从写操作到读操作的依赖关系,但放宽了对执行顺序的要求。通过精确地探索每个“reads-from”等价类中的一个执行路径,避免对等价的执行路径进行重复探索。根据这个算法,作者实现了Nidhugg/rfsc并进行了大量测试,首先,Nidhugg/rfsc在某些情况下比最快的SMC工具慢,但在其他情况下,它在探索执行数量上的性能要么类似,要么更好,并且在某些程序中比其他工具快指数级。其次,在测试的49个程序中,Nidhugg/rfsc在大多数程序中找到了bug,并且在其中一些程序中,它是唯一能够找到bug的工具。通过进一步分析,Nidhugg/rfsc可以正确地识别出多线程程序中的reads-from relation。

\subsubsection{利用模糊测试进行漏洞挖掘}

2019年,Jeong\cite{jeong2019razzer}等人开发了Razzer,通过静态分析(point-to analysis)识别,得到所有可能产生冲突的指令对,就是可能访问同一个地址的属于不同线程的两条指令,并得到所有可能产生冲突的指令对集合。然后进行单线程模糊测试以得到可以使程序执行到特定访存指令的输入。然后进行多线程测试,利用单线程测试得到的输入以及断点插入,使得程序按照规划好的顺序交错执行,然后观察是否会让程序发生异常或者其他错误。相比于Syzkaller,这一方法可以用更少的变异、执行次数找到更多的有效bug。但是Razzer的执行速度相比于Syzkaller更慢,并且利用point-to-analysis可能无法找到所有的冲突指令对。

2020年,Hongxu Chen\cite{chen2020muzz}等人提出了MUZZ,这篇文章首先通过静态分析得到可能会产生冲突的scope,然后对这些scope进行插桩。然后进行fuzzing的过程,利用前面插桩 的代码,得到代码覆盖率,线程上下文,交错顺序三个信息,并根据这些信息进行种子的选择,排序,变异,生成新的测试用例。最后利用T-Sanitize检测漏洞。相比于之前的 工作,MUZZ能够产生出更多和多进程相关的种子,也就是说能够测试更多多线程的用例。 同时相比于AFL等工具能够检测出更多的数据争用缺陷.。

2020年,Meng Xu\cite{xu2020krace}等人提出了KRACE,通过引入一个新的覆盖跟踪指标,别名覆盖率,来捕获并发维度的探索进度。通过这个新的维度,结合一种用于生成、变异、合并多线程 系统调用序列的进化算法,引导fuzzer对线程并发空间进行更有效的探索。相对于Syzkaller,KRACE可以发现更多的branch path。但是也会引入更大的开销。 

2022年,Zu-Ming Jiang\cite{jiang2022context}等人通过引入并发调用对这个新的并发覆盖指标,确保上 下文敏感性,并提出了一种邻接导向突变,生成新的可能的线程交错。这篇文章提出了 CONZZER 框架,并在8个用户级程序和4个内核级文件系统上进行了测试,并发现了95个数据竞争,其中75个是有害的。 

2023年,Jeong等人提出SEGFUZZ\cite{jeong2023segfuzz},将对整个线程的交错空间进行分解,对一小段代码进行探索,然后再将多个小段代码合并再进行探索,并提出交织段覆盖率的概念,再提高发现并发错误的准确性的同时,避免了冗余测试。这很大程度上降低了并发空间探索的复杂度。SEGFUZZ在linux内核中发现了21个新的并发错误,并且这些错误都表现出有害行为。并且相对于 Snowboard和KRACE这些之前的工作能够更快地发现并发错误(平 均快 4.1 倍)。 

2023年,Ming Yuan等人开发了DDRACE\cite{yuan2023ddrace},提出了新的距离度量反馈,RPIP(Race Pair Interaction Profile)反馈,约束反馈,来指导生成新的输入,并利用一种交错优先级 方案来选择更有可能产生并发访问缺陷的输入。相对于Syzkaller,RAZZER,KRACE这些工具,DDRACE可以发现更多的并发访问缺陷,并且可以在更短时间内发现。


\subsection{论文主要研究内容}

对本论文的主要工作进行叙述,包括创新点,所作的实验,所有的可以用来作为工作量的东西都放上去

\subsection{论文的组织方式}

对每一章写了些什么进行叙述,这个可以等到后面写完了再来补充,避免会进行调整



\section{背景知识}



本章对相关的背景知识进行详细的介绍。第一个部分是对并发漏洞的介绍,对常见的并发漏洞模式进行梳理,指出其特点。然后整体介绍通用模糊测试,然后介绍模糊测试究竟是什么,其利用到的一些技术。

\subsection{竞争条件}



并发漏洞的触发往往需要特定的线程交错顺序,由于线程交错的空间极大,常规的编译时检查并不能直接检测出来。为了解决这个问题,很多的静态分析工具被开发出来,专门用于并发漏洞的检测。由于静态分析针对源代码进行分析而无需具体运行程序,因此可以以较快的速度分析并得到结果。