\section{研究计划进度安排及预期目标}

\subsection{进度安排}

第一阶段 2023年6月25日 – 2023年10月28日
和老师进行讨论,并初步决定毕业设计的各个环节步骤,对选题的相关领域的基础知识进行系统性的学习,在这个过程中和老师积极进行讨论。并阅读安全领域的相关文献,这个阶段主要在广度上进行积累,主要对安全相关领域的各个方向进行了解,学习目前的前言技术和用到方法。在老师的指导下学习如何进行科研。

第二阶段 2023年10月29日 – 2024年2月24日
学习模糊测试技术,并在利用模糊测试寻找漏洞这个领域大量阅读文献和综述,对当前这个领域的主要研究进展和当前存在的问题有一个比较清晰的了解。并对模糊测试技术在这个领域的应用和改变有更加深刻的认识。

第三阶段 2024年2月25日 – 2024年3月25日
和指导老师进一步讨论,听取指导老师的意见,集中阅读选题相关领域的近几年文献,并撰写文献综述、开题报告、外文翻译。根据当前领域的最新研究,对相关的工作进行复现和测试,提出自己的改进方案和解决方法。对选题以及可行性进行分析。

第四阶段 2024年3月26日 – 2024年4月12日
对当前主要参考的工作进一步分析,寻找当前工作的缺陷并分析导致这种缺陷的原因,利用之前积累的当前领域的最新进展,并根据自己的观察,提出自己的改进方案和思路。并做一些简单的实验验证这些想法的合理性。

第五阶段 2024年4月13日 – 2024年4月25日
基于之前的观察和自己提出的方案,设计系统的架构并构建系统。

第六阶段 2024年4月26日 – 2024年5月5日
对构建好的系统进行实验,测试系统对并发访问漏洞的检测效果,对测试的结果以及新发现的漏洞进行分析。将当前系统的测试结果和其他工作的测试结果进行对比,分析优势以及不足指出,指出应该在哪些方面进行进一步的改进。

第七阶段 2024年5月6日 – 2024年5月20日
撰写毕业论文和相关材料,并和指导老师进行讨论沟通,对论文进行修改,提交最终版本的论文。

\subsection{预期目标}

基本目标:

\begin{enumerate}
\item 成功构建出系统,并能够在检测并发访问漏洞上有较好的效果,在模糊测试的过程中同时兼顾效率和表现,在不给模糊测试带来大的开销的前提下,能够有效检测出并发访问漏洞。
\item 和之前的工作相比,能够在整体或者特定方面有比较大的提升。
\item 在项目的进行过程中,学习科研的基本方法,在项目完成之后能够独立进行科研探索。
\item 撰写最终的科研论文,在整个项目的进行过程中和指导老师积极进行讨论,听取老师的建议。
\end{enumerate}

